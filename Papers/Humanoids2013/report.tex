Problem
Research
Problem
Hypothesis
Materials
-Given
-Procured
-Developed
Procedure
-Typical
-Developed
Results
-Correctness
-Performance
Conclusion
-Future Work

Problem

For the DARPA robotics challenge, we are required to operate an unknown valve in an unknown environment.  In this particular case, the knowns are that the valve will be toroidal in shape and require some non-negligible amount of torque to operate it (i.e. twist it about its axis).  We are given a humanoid robot to achieve the task, although in our first hardware revision, wheels supplant legs.

In the task, we are given essentially boundless control over the robot - autonomous methods are as appropriate as full teleoperation.  In this circumstance, the question for roboticists becomes: What level of autonomy is optimal?  In this paper, we will present our findings for achieving this task at various levels of autonomy.

Research

The various levels of autonomy that we are considering are three. Full teleoperation occupies the zero autonomy level.  Entirely onboard sensor driven behaviors occupy the full autonomy level. In between is a nebulous zone described as "semi-autonomy" or "assisted teleoperation."  In this section, we will briefly describe research accomplished in each of these levels that directly relate to the aforementioned problem.

Full Teleoperation
Full autonomy
Semi-autonomy

Hypothesis

Given the choices above, and previous experiences, we predicted that teleoperation would accomplish the task, but be tedious due to many precautionary limits.  Fully autonomy may take several days of testing to get right - in one particular environment.  Semi-autonomy may allow us to gouge out the difficult portions of implementing "robustness" in autonomous routines.

Materials

To test our hypothesis, we gathered our wheeled humanoid robot, a set of variously sized steering wheels, and a wifi station to connect our computers to the robot.  The robot is...

Procedure

To measure the performance and correctness of the system, we recorded the following measurements:
Ability to Execute
Time to completion
Operator Attention Time

We concocted the following scenarios:
Fully autonomous driving to steering wheel and fully autonomous operation of the steering wheel
Fully autonomous driving to steering wheel and fully teleoperated operation of the steering wheel
Fully teleoperated driving to steering wheel and fully autonomous operation of the steering wheel
Fully teleoperated driving to steering wheel and fully teleoperated operation of the steering wheel
Semi-autonomous driving to steering wheel and semi-autonomous operation of the steering wheel

We measure the ability to execute with one point allocated for a "decent" alignment position of the robot to the steering wheel, one point for successful gripping of the steering wheel, and one point for successful manipulation of the steering wheel.

We measure the time to completion as the time it took to finish the alignment, time to grab the wheel after aligning, and time to turn the wheel back and forth thirty degrees in both directions.

We measure the operator attention time as the amount of time the operator spends attempting to send commands to the robot.  The times are recorded for each stage of aligning, gripping, and manipulating.

A set of novice individuals and a set of experienced individuals were recruited for training.  Specifically, 3 users who were adept with the system and 3 users who were new to the system were engaged in trials of each mode of operation.  The order in which the user attempted the methods was randomly assigned, however, the order of events that included fully teleoperated modes were the ones assumed to include skill learning, as opposed to the semi-autonomous method.

Methods

Our autonomous steering wheel detection routine was devised in the following manner.